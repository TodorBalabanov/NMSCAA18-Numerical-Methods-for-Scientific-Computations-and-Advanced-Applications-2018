\documentclass{beamer}

\mode<presentation> {
	\usetheme{Berlin}
}

\title[Numerical Methods for Scientific Computations and Advanced Applications, Hissarya, Bulgaria]{
	Artificial Neural Network Activation Function Optimization with Genetic Algorithms
}

\author{Ivan Blagoev, Janeta Sevova, Kolyu Kolev}

\date{28-31.V.2018}

\institute[IICT-BAS, NMSCAA'18] {
	Institute of Information and Communication Technologies \\ 
	Bulgarian Academy of Sciences \\
	\medskip
	\textit{i.blagoev@iit.bas.bg}
}

\begin{document}

\begin{frame}
\titlepage
\end{frame}

\begin{frame}
\frametitle{Overview}
\tableofcontents
\end{frame}

\section{Introduction}

\subsection{Neuron Activation Function}

\begin{frame}
\frametitle{List of Commonly Used Activation Functions}
\begin{figure}[h]
  \centering
  \includegraphics[width=0.5\linewidth]{fig01}
\label{fig:01}
\end{figure}
\end{frame}

\section{Experiments and Results}

\section{Conclusions}

\subsection{Advantages and Disadvantages}

\begin{frame}
\frametitle{Conclusions}
\begin{itemize}
  \item Findings for alternative activation functions can lead to speed-up of ANN training.
\end{itemize}
\end{frame}

\subsection{Q\&A}

\begin{frame}
\frametitle{Questions and Answers}
\center \huge{Thank you for the attention!}
\end{frame}

\end{document}
